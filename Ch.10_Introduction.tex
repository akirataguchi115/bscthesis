\chapter{Johdanto\label{intro}}

Koneoppimisen käyttötarkoitusten määrä kasvaa vuosi vuodelta suuremmaksi. Tätä teknologiaa voidaan hyödyntää muun muassa ihmisten puhuman kielen käsittelyy. Luonnollisen kielen käsittely (eng. Natural Language Processing, NLP) on alati kasvavassa kuluttajakäytössä johtuen laskentatehon kasvusta, suurien tietomäärien saatavuudesta, onnistuneiden koneoppimismetodien kehittämisestä sekä laajemmasta ihmiskielen ymmärryksestä ja sen käytöstä eri konteksteissa \citep{doi:10.1126/science.aaa8685}.

Luonnollisen kielen käsittely on kohdennetun mainonnan keskiössä. Viesti ystävälle mainoskohdennetussa viestipalvelussa antaa työstettävän datan NLP-mallille: ``Mikä elokuva meidän pitäisi katsoa viikonloppuna? `` NLP-mallin avulla automaattinen mainostaja ymmärtää mainostaa kyseiselle käyttäjälle miltei välittömästi sarjalippuja mainostavasta elokuvateatterista, suoratoistopalvelua tai mainostavaa aktiviteettikeskusta kyseiselle viikonlopulle.

Tässä tutkielmassa tarkastellaan NLP-hyökkäysten käyttökohteita. Tähän kuuluu hyökkäystaksonomia, puolustusmetodit sekä NLP-mallien sekä niihin kohdistuvien hyökkäysten tulevaisuus. Koska NLP-mallit ovat eksponentiaalisessa nousussa kuluttajakäytössä, on tärkeää spekuloida mahdollisia kehityksiä koneoppimisessa sekä tästä syntyviä haavoittuvuuksia. \citep{yang-etal-2021-rethinking}. \citep{boucher2021bad}