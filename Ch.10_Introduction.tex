\chapter{Johdanto\label{intro}}

Koneoppimisen käyttötarkoitusten määrä kasvaa vuosi vuodelta suuremmaksi. Tätä teknologiaa voidaan hyödyntää muun muassa ihmisten puhuman kielen käsittelyy. Luonnollisen kielen käsittely (eng. Natural Language Processing, NLP) on alati kasvavassa kuluttajakäytössä johtuen seuraavista syistä:
\begin{itemize}
  \item laskentatehon kasvu
  \item suurien tietomäärien saatavuus
  \item onnistuneiden koneoppimismenetelmien kehitys
  \item sekä laajempi ihmiskielen ymmärrys ja sen käyttö eri\\ konteksteissa \citep{doi:10.1126/science.aaa8685}.
\end{itemize}

Luonnollisen kielen käsittelyä voidaan hyödyntää kohdennetussa mainonnassa. Analysoimalla NLP-mallin avulla esimerkiksi käyttäjien lähettämiä viestejä toisilleen, voidaan saada selville tuote, jota kannattaa mainostaa mainoskohdennetulle yksilöllle. Viesti ystävälle mainoskohdennetussa viestipalvelussa antaa työstettävän datan NLP-mallille: ``Mikä elokuva meidän pitäisi katsoa viikonloppuna? `` NLP-mallin avulla automaattinen mainostaja ymmärtää mainostaa kyseiselle käyttäjälle miltei välittömästi sarjalippuja mainostavasta elokuvateatterista, suoratoistopalvelua tai mainostavaa aktiviteettikeskusta kyseiselle viikonlopulle. Tämän rahanarvoisen tarpeen löytäminen datasta automaation avulla edellyttää kaikkia neljää aikaisemmin mainittua teknologista edistystä kultakin osa-alueelta.

Kaikkien neljän osa-alueen kehittyminen mahdollistaa luonnollisen kielen käsittelyn yleistymisen. Ihmiskielen ymmärtäminen tietokoneen tasolla on kehittynyt huomattavasti, kun ihmisen käyttämää kieltä, virkkeitä ja sanoja on alettu pilkkomaan helpommin ymmärrettäviksi paloiksi \citep{https://doi.org/10.1002/aris.1440370103}. Jotta luonnollisen kielen käsittelyn malli olisi rakennettu älykkäästi, tarvitsemme edistyneitä koneoppimismetodeita. Tämä on tullut kehityksen saatossa mahdolliseksi \citep{jordan2015machine}. Koska datan määrä on kasvanut ja dataa on helpompaa hankkia \citep{gopalakrishnan2018deep}, pystymme kouluttamaan mallin toimimaan mahdollisimman monessa eri tilanteessa. Laskentatehon huomattava kasvu vuosien mittaan \citep{moore1965cramming} on alkanut mahdollistaa suurempien datamäärän käsittelyä kuin aikaisemmin.

Tässä tutkielmassa tarkastellaan NLP-hyökkäysten käyttökohteita. Tähän kuuluu hyökkäystaksonomia, puolustusmenetelmät sekä NLP-mallien sekä niihin kohdistuvien hyökkäysten tulevaisuus. Hyökkäystaksonomiassa käymme läpi erilaisia tapoja hyökätä NLP\- malleja vastaan, hyökkäysten tarkoituksiin ja onnistumisen todennäköisyyksiin. Puolustusmenetelmät ovat tärkeässä osassa, jotta haavoittuvuuteen kohdistuvat yritykset saavat ohjeita vahingon mitigointiin ja ennaltaehkäisyyn. On tärkeää myös spekuloida mahdollisia kehityksiä koneoppimisessa sekä tästä syntyviä haavoittuvuuksia. Lopuksi käymme läpi mahdollisia luonnollisen kielen käyttökohteita tulevaisuudessa sekä näistä aiheutuvia seurauksia eri osa-alueisiin sekä akateemisella että kaupallisella puolella.