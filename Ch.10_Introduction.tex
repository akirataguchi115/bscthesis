\chapter{Sisällys\label{intro}}

Ohjelmistojen hyökkäysrajapinta-ala kasvaa jatkuvasti. Osa haavoittuvaisuuksista korjataan heti havainnoinnin jälkeen, osa mitigoidaan ja osan vaikutusalue on manifestoituu vasta tulevaisuudessa . Luonnollisen kielen prosessointi (eng. Natural Language Processing, NLP) on osoittautunut hyväksi hyökkäysrajapinnaksi tätä teknologiaa hyödyntäviä osapuolia vastaan \citep{boucher2021bad}. NLP-järjestelmät on tehty tulkitsemaan ihmisen luonnollista kieltä. Tämän kielen konekääntäminen aloitettiin jo vuonna 1949.

Tässä tutkielmassa tarkastellaan NLP-hyökkäysten käyttökohteita. Tähän kuuluu oleellisen historian esittely, hyökkäystaksonomia sekä puolustusmetodit. On tärkeää ymmärtää luonnollisen kielen prosessoinnin tarkoitus, jotta voidaan syventyä hyökkäysiä mahdollistaviin ongelmiin sekä näiden ratkaisemiseen.