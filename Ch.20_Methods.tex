\chapter{Tarve luonnollisen kielen käsittelylle\label{methods}}

Ehdotukset kielten välisten sanojen välittämisestä koodeilla esitettiin 1700-luvulla Leibnizin ja Descartesin johdolla. Vuonna 1957 Georgetown-IBM-kokeen tekijät väittivät 3-5 vuoden jälkeen konekääntämisen olevan ratkaistu ongelma.

Tarve konekääntämiselle kumpuaa tietokoneen vajeesta ymmärtää ihmisen puhumaa kieltä. Ohjelmoinnissa tämän käännöksen tekee ihminen kutsuessaan esimerkiksi python-tulkilla "print("Hello world")". Käännöksen tapahtuminen tietokonepuolella tuottaa mielekkäämpiä ongelmia. Komento "Tulosta syntymäpäiväni" saattaa koneoppimismallista riippuen tulostaa näytölle merkkijonon "1.1.1970" tai fyysisen kuvan syntymästäsi lähikirjastosi tulostimeen.

NLP-hyökkäyksissä keskiössä on juuri tämän tulkitsemisen vaikeuden hyväksikäyttäminen pahansuopiin tarkoituksiin.