\chapter{Tekstin automaattinen luokitus\label{methods}}

NLP-luokittimia, eli NLP-malleja käytetään tapauksiin, joissa on tehokkaampaa korvata ihmisen manuaalinen tekemä tarkastustyö. Näihin tapauksiin kuuluvat muun muassa roskapostin tunnistus sekä vihapuheen tunnistus sosiaalisesta mediasta. Esimerkiksi Twitterissä käytetään NLP-malleja tunnistamaan sopimatonta sisältä twiiteistä \citep{twitternlp}.

Tämä kaikki tarkastustyö voitaisiin tehdä manuaalisesti käsin, mutta tarkastettavan sisällön määrän vuoksi tämä ei ole käytännössä mahdollista. Tietoteknistaitoinen ihminen pystyisi tarkastamaan vastaanotetusta sähköpostista, mikäli kyseinen sähköposti olisi esimerkiksi kalasteluroskapostia. Koska roskapostia lähetetään automaattisesti jokaiseen olemassa olevaan sähköpostiosoitteeseen päivittäin, menisi roskapostien tunnistamiseen ihmiseltä liian kauan aikaa päivittäin. Tämän takia useimmissa sähköpostiohjelmissa tulee mukana automaattisesti roskapostia suodattava NLP-malli, joka päästää läpi vain sähköpostit, joista NLP-malli ei ole varma, onko se roskapostia. Perehdytään seuraavassa kappaleessa tarkemmin tämän suodattimen ohitukseen. Tämä haavoittuvaisuus on läsnä myös muissa sovelluksissa, joissa käytetään NLP-mallia.