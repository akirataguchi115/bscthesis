\chapter{Hyökkäystaksonomia\label{methods}}
Käymme seuraavaksi läpi neljä erilaista huomaamatonta hyökkäysmetodia. Nämä hyökkäykset eivät siis näy visuaalisesti ihmiskäyttäjälle näyttöpäätteellä erilaisina verrattuna viattomaan tekstiin. Tarkemmin keskitymme Unicoden ja muiden enkoodausmenetelmien hyväksikäyttämiseen NLP-malleja vastaan.

\section{Näkymättömät merkit}
Näkymättömät merkit vaikuttavat tietokoneen NLP-mallin ymmärtämään kontekstiin. Esimerkki tällaisesta on nollatilavuuden välilyönti -merkki, jonka Unicode merkintä on \texttt{U+200B}. Tällä merkillä voimme esimerkiksi vaikuttaa pelichattiin lähetettävän toksissuodatettavan merkkijonoon "olet huono" niin, että merkkijono menisi NLP-mallin läpi chätistä. Merkkijono \texttt{olU+200Bet huU+200Bono} saattaisi mennä läpi chatin suodattimesta, mutta vastapuolelle viesti olisi edelleen \texttt{olet huono}.

Kontekstin poistamisen lisäksi näkymättömillä merkeillä voidaan myös tuoda ja syrjäyttää konteksteja toisilla.\\
\texttt{Mikä pyhäinhäväistyksen rakennus!\\
Miten onnistuit tekemään tämän näin laiskasti?} -tekstin negatiivisuus voidaan syrjäyttää positiivisuudella syöttämällä NLP-mallille sen sijaan teksti \\
\texttt{Mikä py\textcolor{red}{U+200B}häinhäv\textcolor{red}{U+200B}äisty\textcolor{red}{U+200B}ksen\textcolor{red}{U+200B} rakennus!\\
Miten onnistuit tekemään tämän \textcolor{red}{U+200B}nä\textcolor{red}{U+200B}in la\textcolor{red}{U+200B}iskas\textcolor{red}{U+200B}ti?}.

\section{Homoglyfit}
Homoglyyfihyökkäykset NLP-malleja vastaan pohjautuvat pahantahtoisten merkkien virallisten esitysmuotojen näyttävän hyväntahtoisten merkkien virallisilta esityksiltä. Joissain kielissä tekstin merkitys muuttuu täysin yhden merkin vaihtuessa. Esimerkkinä homoglyyfistä on \texttt{A $\rightarrow$ A}, missä viimeinen kirjain on todellisuudessa kyrillinen kirjain A. Näkymättömien merkkien lailla homoglyyfihyökkäyksen toteutus riippuu ympäristön fontista.

\section{Uudelleenjärjestelyt}
Uudelleenjärjestelyhyökkäys pohjautuu näennäisen tekstin uudelleenjärjestämiseen pahantahtoisesti. Pankkitilinumeron \texttt{1234567} pystyy esimerkiksi vaihtamaan kaksisuuntaisella-algoritmilla tilinumeroksi \texttt{7654321} maksajan huomaamatta mitään.

\section{Poistatukset}
Poistatushyökkäyksiä on vaikeampi toteuttaa aikaisempiin metodeihin verrattuna. Tämä johtuu useimpien käyttöjärjestelmien estosta kopioida poistatusta sisältävää tekstiä leikepöydälle.