
\appendix{Tutkielmapohjan käyttöohjeet}
\label{appendix:instructions_finnish}

\section{Ensiaslkeleet}

\texttt{HY-CS-main.tex} tiedosto sisältää viisi askelta STEPS 0--5. Alla on kuvattu, mitä nämä askeleet tarkoittavat ja miten niitä seuraamalla luot pohjan tutkielmallesi.
\vspace{0.5cm}

\textbf{STEP 0 -- Kopioi tutkielmapohja}

\begin{itemize}
\item Hae tutkielmapohja uuteen Overleaf-projektiin. Tämä käy helpoiten seuraavasti:
\begin{itemize}
    \item Lataa Latex-pohjan zip-tiedosto koulutusohjelman sivuilta.
    \item Mene osoitteeseen \url{www.overleaf.com/edu/helsinki} ja kirjaudu Overleafiin yli\-opiston tunnuksillasi.
    \item Overleafissa (\url{https://www.overleaf.com/project}), klikkaa ``New Project'' and ``Upload Project''.
    \item Valitse lataamasi tutkielmapohjan zip-tiedosto.
    \item Nyt voit lähteä kirjoittamaan tutkielmaasi suoraan pohjaan, voit aloittaa esim. vaihtamalla projektin nimen.
\end{itemize}
\end{itemize}


{\textbf{STEP 1 -- BSc vai MSc tutkielma?}}
\begin{enumerate}
\item Valitse (tiedostossa \texttt{HY-CS-main.tex}) oletko tekemässä BSc (tkt) vai MSc (csm tietojenkäsittely) tutkielmaa.
\item Valitse kieli jolla kirjoitat tutkielman: \texttt{finnish}, \texttt{english} tai \texttt{swedish}.
\item Jos olet kirjoittamassa maisterintutkielmaa, valitse linja/opintosuunta.
\end{enumerate}


{\textbf{STEP 2 -- Aseta henkilökohtaiset tietosi}}

\begin{enumerate}
\item Kirjoita alustava otsikko tutkielmallesi: \texttt{\textbackslash title\{\}}.
\item Kirjoita oma nimesi kohtaan \texttt{\textbackslash author\{\}}.
\item Lisää ohjaajien nimet \texttt{\textbackslash supervisors\{\}}.
\item Määrittele avainsanat \texttt{\textbackslash keywords\{\}}.
\item Määritä tutkielmasi ACM luokittelutermit \texttt{\textbackslash classification\{\}}. Ks. lisätietoa: \url{https://dl.acm.org/ccs}.
\end{enumerate}

{\textbf{STEP 3 -- Kirjoita tiivistelmä}}

%\begin{itemize}
%\item 
Voit kirjoittaa tiivistelmän (koko tiivistelmäsivu) eri kielillä \texttt{otherlanguages}-ym\-pä\-ris\-tön avulla. Alla esimerkki jolla kirjoitat englanninkielisen tiivistelmän muulla kuin englannin kielellä kirjoitettuun tutkielmaan:

\begin{verbatim}
\begin{otherlanguage}{english} 
\begin{abstract}
Your abstract text goes here. 
\end{abstract} 
\end{otherlanguage}
\end{verbatim}

%\end{itemize}

{\textbf{STEP 4 -- Kirjoita tutkielma}}

\begin{enumerate}
\item Kirjoittamisesta Latexilla löydät hieman ohjeita alempaa.
\item Poista tämä liite ja muu ohjeistus tutkielmastasi, esim. kommentoimalla.
\end{enumerate}

{\textbf{STEP 5 -- Aseta kirjallisuuslähdeluettelon tyyli}}

\begin{itemize}
\item Oletustyylin tekijä-vuosi, eli (Einstein, 1905), voit vaihtaa viittaustyylin (tiedostossa \texttt{HY-CS-main.tex}) helposti (eri mallit kommentoituna) esim. numeroituun [1], tai aakkostyyliin [Ein05].
Lisää ohjeita liittyen viittaustyylin säätämiseen {Bib}\TeX issä löytyy verkosta: \url{https://ctan.org/pkg/biblatex}
\item Sovi käytettävä tyyli ohjaajasi kanssa. 
\end{itemize}

\section{Kirjallisuusviitteet Latexissa}

Kirjallisuuslähteet ylläpidetään erillisessä .bib-tiedostossa. Tässä tutkielmapohjassa käy\-te\-tyt kirjallisuuslähteet, joista esimerkkejä kuvassa~\ref{bibexamples-fi}, löytyvät tiedostosta\newline \texttt{bibliography.bib}.

\begin{figure}[ht]
    \centering
    \begin{scriptsize}
\begin{verbatim}

@article{einstein,
    author =       "Albert Einstein",
    title =        "{Zur Elektrodynamik bewegter K{\"o}rper}. ({German})
        [{On} the electrodynamics of moving bodies]",
    journal =      "Annalen der Physik",
    volume =       "322",
    number =       "10",
    pages =        "891--921",
    year =         "1905",
    DOI =          "http://dx.doi.org/10.1002/andp.19053221004"
}

@book{latexcompanion,
    author    = "Michel Goossens and Frank Mittelbach and Alexander Samarin",
    title     = "The \LaTeX\ Companion",
    year      = "1993",
    publisher = "Addison-Wesley",
    address   = "Reading, Massachusetts"
}

@book{knuth99,
    author    = "Donald E. Knuth",
    title     = "Digital Typography",
    year      = "1999",
    publisher = "The Center for the Study of Language and Information",
    series    = "CLSI Lecture Notes (78)"
}
\end{verbatim}
\end{scriptsize}
    \caption{Esimerkkejä kirjallisuuslähteiden kuvaamisesta .bib-tiedostossa.}
    \label{bibexamples-fi}
\end{figure}

Viitteet kirjallisuuslähteisiin muodostetaan komennolla \texttt{\textbackslash citep\{einstein\}}, josta generoituu tekstiin valitun viittaustyylin mukaisesti muotoiltu viite \citep{einstein}, tai \texttt{\textbackslash citep\{latexcompanion,knuth99\}}, josta tekstiin puolestaan generoituu \citep{latexcompanion,knuth99}. 
Voit esimerkiksi kirjoittaa \citep{einstein} viitataksesi julkaisuun, jonka on kirjoittanut \citeauthor{einstein} vuonna \citeyear{einstein}, kun vain lähde \citet{einstein} on oikein lisättynä kirjallisuuslähdetiedostossa (katso miltä nämä näyttävät Latex lähdekoodissa).

Tekstissä viitatut kirjallisuuslähteet tulevat automaattisesti viiteluetteloon. Kirjallisuuslähteiden tietojen oikeellisuus ja yhdenmukaisuus .bib-tiedostossa vaikuttavat luonnollisesti siihen, miten tiedot tutkielmassa näyttäytyvät. Tämä on syytä huomioida, sillä esim.\ verkosta valmiiksi {Bib\TeX} muodossa löytyvien tietojen täydellisyyten tai samanmuotoisuuteen ei pidä sokeasti luottaa.  


Keskustele viittaustyylin valinnasta ohjaajan kanssa. 
%Joitain vaihtoehtoja on osoitteessa:\\ 
%\url{https://www.overleaf.com/learn/latex/Biblatex_bibliography_styles}.
%\url{https://www.sharelatex.com/learn/Bibtex_bibliography_styles}.

\section{Joitain ohjeita Latexilla kirjoittamiseen}

Seuraavassa on joitain ohjeita tämän tutkielmapohjan käyttöön maisterintutkielmassa. Kirjoittamisohjeita löytyy useasta eri lähteestä. Voit esimerkiksi tutustua kandidaatintutkielman ohjeisiin. 
Ohjaajan kanssa on hyvä keskustella aikaisessa vaiheessa työn rakenteesta.

Yksityiskohtaisia ohjeita Latexin käyttämäsestä saa parhaiten hakemalla verkosta, esim. haku englanniksi "Overleaf table positioning" tuottaa oletettavasti aika toimivan vastauksen.

\section{Kuvat}
Kuva~\ref{fig:logo-fi} toimii esimerkkinä kuvan lisäämisestä työhön (katso tarkemmin mallia Latex lähdekoodista). Muista myös viitata jokaiseen kuvaan tekstissä. 

\begin{figure}[ht] % remove [h!] for automatic placement, which is probably better for a thesis with more text on page
\centering 
\includegraphics[width=0.3\textwidth]{template/figures/HY-logo-ml.png}
\caption{Helsingin yliopiston logo matemaattis-luonnontieteellisen tiedekunnan värein.\label{fig:logo-fi}}
\end{figure}

\newpage % just to keep the table on the same page with the short piece of text
\section{Taulukot}

Taulukossa~\ref{table:results-fi} on esimerkki kokeellisten tulosten raportoinnista taulukkona. Muista myös viitata jokaiseen taulukkoon tekstissä.
\begin{table}[ht]
\centering
\caption{Kokeelliset tulokset.\label{table:results-fi}}
\begin{tabular}{l||l c r} 
Koe & 1 & 2 & 3 \\ 
\hline \hline 
$A$ & 2.5 & 4.7 & -11 \\
$B$ & 8.0 & -3.7 & 12.6 \\
$A+B$ & 10.5 & 1.0 & 1.6 \\
\hline
%
\end{tabular}
\end{table}

