\chapter{Yhteenveto\label{conclusions}}

Luonnolisen kielen käsittely on muovautunut tärkeäksi osaksi tietokoneteollisuutta. Koneoppimisen avulla kuluttajan käyttämästä ihmiskielestä saadaan käyttöön rahanarvoista mainontatietoa, jota yritys pystyy käyttämään joko itse tai myymään sen eniten tarjoavalle taholle. Tarve ihmiskielen koneelliseen ymmärrykseen on tuonut mukanaan kiinnostuksen lisäksi tietoturvatietoisuutta aiheesta.

Kävimme läpi tässä tutkimuksessa tekijöitä luonnollisen kielen käsittelyn kehitykseen, joita ovat laskentateho, tietomäärä, koneoppiminen sekä ihmiskielen ymmärrys. Kävimme läpi hyökkäyspinta-alan ja puolustusmahdollisuudet NLP-malleista johtuvia tietoturvauhkia vastaan. Lopuksi käytiin myös läpi tekstipohjaisten vastakkaishyökkäysten tulevaisuutta NLP-malleja vastaan.

Koneoppiminen on todennäköisesti vasta kehitysvaiheen alkupuolella. Jo nyt näkemämme vastakkaishyökkäykset osoittavat useampia haavoittuvaisuuksia, kuin mitä vastaan pystymme puolustautumaan. Selvästi suurin syy tälle on tekoälyn valjastettu voima, jota emme vielä pysty hallitsemaan kaikissa reunatapauksissa.