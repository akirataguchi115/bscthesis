\chapter{Yhteenveto\label{conclusions}}

Kävimme läpi tässä tutkimuksessa tekijöitä luonnollisen kielen käsittelyn kehitykseen, joita ovat laskentateho, tietomäärä, koneoppiminen sekä ihmiskielen ymmärrys. Kävimme läpi hyökkäyspinta-alan ja puolustusmahdollisuudet NLP-luokittimista johtuvia tietoturvauhkia vastaan. Lopuksi käytiin myös läpi tekstipohjaisten vastakkaishyökkäysten tulevaisuutta NLP-luokittimia vastaan.

Kuten aikaisemmin mainittiin, neljä mahdollistajaa luonnollisen kielen käsittelyyn kuluttajakäytössä ovat laskentatehon kasvu, suurien tietomäärien saatavuus, onnistuneiden koneoppimismenetelmien kehittäminen sekä laajempi ihmiskielen ymmärrys ja käyttö eri konteksteissa. NLP-luokittimien mahdollistajien kehittyessä arvaamattomasti, on loogista tutkia myös NLP-hyökkäysten tulevaisuutta. Vastakkaishyökkäysten motiivit muovautuvat siis ajan myötä ja kasvattavat tahtomattaan näin hyökkäystyyppien määrää. 

Hyökkäystyypit laajentuvat tulevaisuudessa eri formaatteihin. Koneoppimisen kukoistaessa voidaan NLP-luokittimia soveltaa tiedon ääni -tai videoformaatteihin. Tämä antaa puolestaan mahdollisuuden vastakkaishyökätä kyseiseen koneoppimismallia vastaan. Formaattien sisältäkin löytyy erinäisiä hyökkäystyyppejä. Esimerkiksi ääniformaateissa käytetään kuhunkin käyttötarkoitukseen sopivaa enkoodausta. Ei siis riitä, että hyökättävää ja puolustettavaa tulee uusien formaattien myötä, sillä formaattien sisälläkin tulee tapahtumaan jatkuvasti huomattavaa kehitystä.

Lisäksi haavoittuvuuksien löytö ruokkii itse itseään. Ensimmäisten vastakkaishyökkäysten kohdistuessa uuteen tietoformaattiin, syntyy tarve puolustukseen tätä vastaan. Toteutuksesta riippuen puolustusmenetelmän selvittäminen saattaa avata uusia ovia, jotka hyödyttävät hyökkääjiä. Usein haavoittuvuuden tarkastelu vastakkaishyökkäyksissä avaa enemmän mahdollisuuksia uusille hyökkäyksille kuin vanhojen hyökkäysten puolustuksille.

Luonnolisen kielen käsittely on muovautunut tärkeäksi osaksi tietokoneteollisuutta. Koneoppimisen avulla kuluttajan käyttämästä ihmiskielestä saadaan käyttöön rahanarvoista mainontatietoa, jota yritys pystyy käyttämään joko itse tai myymään sen eniten tarjoavalle taholle. Rahanarvoisen hyödyn lisäksi luonnollisen kielen käsittely tarjoaa myös yleishyödyllisiä ratkaisuja, kuten vihapuheen esto ja roskapostisuodattimet. Tarve ihmiskielen koneelliseen ymmärrykseen ja haluun valjastaa kestävästi sen hyödyt ovat tuoneet mukanaan kiinnostuksen luonnollisen kielen käsittelyn tietoturvaan.