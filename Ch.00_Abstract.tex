% \begin{abstract}{finnish}

% Tämä dokumentti on tarkoitettu Helsingin yliopiston tietojenkäsittelytieteen osaston opin\-näyt\-teiden ja harjoitustöiden ulkoasun ohjeeksi ja mallipohjaksi. Ohje soveltuu kanditutkielmiin, ohjelmistotuotantoprojekteihin, seminaareihin ja maisterintutkielmiin. Tämän ohjeen lisäksi on seurattava niitä ohjeita, jotka opastavat valitsemaan kuhunkin osioon tieteellisesti kiinnostavaa, syvällisesti pohdittua sisältöä.


% Työn aihe luokitellaan  
% ACM Computing Classification System (CCS) mukaisesti, 
% ks.\ \url{https://dl.acm.org/ccs}. 
% Käytä muutamaa termipolkua (1--3), jotka alkavat juuritermistä ja joissa polun tarkentuvat luokat erotetaan toisistaan oikealle osoittavalla nuolella.

% \end{abstract}

\begin{otherlanguage}{finnish}
\begin{abstract}
Tässä tutkimuksessa käsitellään tekstipohjaisia vastakkaishyökkäyksiä NLP-malleja vastaan. Tutustutaan hyökkäystaksonomiaan sekä puolustusmetodeihin yleisimmillä osa-alueilla. Tarkastellaan myös vastakkaishyökkäysten tulevaisuutta teknologian ja yhteiskunnallisten rakenteiden kehittyessä.

Tullaan tunnistamaan vastakkaishyökkäysten rooli nykyteknologiassa, sekä syyt merkittävyyteen koneoppimismallien käytössä. Huomaamaan myös hyökkäyspinta-alan kattavan laajemman alueen puolustuspinta-alaan verrattuna vastakkaishyökkäyksissä. Huomataan syyn olevan jo-valjastettu, mutta reunatapauksissa hallitsematon tekoälyn voima.
\end{abstract}
\end{otherlanguage}
