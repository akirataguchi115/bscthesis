% \begin{abstract}{finnish}

% Tämä dokumentti on tarkoitettu Helsingin yliopiston tietojenkäsittelytieteen osaston opin\-näyt\-teiden ja harjoitustöiden ulkoasun ohjeeksi ja mallipohjaksi. Ohje soveltuu kanditutkielmiin, ohjelmistotuotantoprojekteihin, seminaareihin ja maisterintutkielmiin. Tämän ohjeen lisäksi on seurattava niitä ohjeita, jotka opastavat valitsemaan kuhunkin osioon tieteellisesti kiinnostavaa, syvällisesti pohdittua sisältöä.


% Työn aihe luokitellaan  
% ACM Computing Classification System (CCS) mukaisesti, 
% ks.\ \url{https://dl.acm.org/ccs}. 
% Käytä muutamaa termipolkua (1--3), jotka alkavat juuritermistä ja joissa polun tarkentuvat luokat erotetaan toisistaan oikealle osoittavalla nuolella.

% \end{abstract}

\begin{otherlanguage}{finnish}
\begin{abstract}
Tässä tutkielmassa käsitellään tekstin automaattista luokitusta luonnollisen kielen käsittelyyn pohjautuvilla metodeilla, vastakkaishyökkäyksiä näitä luokittimia vastaan sekä yleisimpiä puolustusmetodeita vastakkaishyökkäyksille. Painopisteenä on NLP-luokittimien käyttökohteet ja luokittimiin kohdistuvien hyökkäysten tyypit. Tarkoituksena on luoda aihepiirin julkaisujen perusteella kokonaiskuva NLP-luokittimista ja näiden haavoittuvaisuuksista, joita hyväksikäytetään vastakkaishyökkäyksissä.

Tullaan huomaamaan, että tekstin automaattisella luokituksella on paljon käyttökohteita, joilla on merkittävät taloudelliset vaikutukset yhteiskuntaan. Luokittimiin kohdistuvat vastakkaishyökkäykset perustuvatu useammissa tapauksissa tekstin ladontaan. Lisäksi vastakkaishyökkäyksiä vastaan kohdistuvat puolustusmenetelmät pyrkivät paikkaamaan alkuperäisessä luokittimessa kohdan, jonka haavoittuvaisuutta hyväksikäytettiin alkuperäisessä hyökkäyksessä. Sykli jatkuu seuraavan vastakkaishyökkäyksen kohdistuessa puolustusmenetelmän tuottamaan luokittimeen.

\end{abstract}
\end{otherlanguage}
