% \begin{abstract}{finnish}

% Tämä dokumentti on tarkoitettu Helsingin yliopiston tietojenkäsittelytieteen osaston opin\-näyt\-teiden ja harjoitustöiden ulkoasun ohjeeksi ja mallipohjaksi. Ohje soveltuu kanditutkielmiin, ohjelmistotuotantoprojekteihin, seminaareihin ja maisterintutkielmiin. Tämän ohjeen lisäksi on seurattava niitä ohjeita, jotka opastavat valitsemaan kuhunkin osioon tieteellisesti kiinnostavaa, syvällisesti pohdittua sisältöä.


% Työn aihe luokitellaan  
% ACM Computing Classification System (CCS) mukaisesti, 
% ks.\ \url{https://dl.acm.org/ccs}. 
% Käytä muutamaa termipolkua (1--3), jotka alkavat juuritermistä ja joissa polun tarkentuvat luokat erotetaan toisistaan oikealle osoittavalla nuolella.

% \end{abstract}

\begin{otherlanguage}{finnish}
\begin{abstract}
Luonnolisen kielen käsittely on kätevä työkalu käsittelemään ihmisten puhumaa kieltä tietokonemaailmassa.

Luonnollisen kielen käsittely on kuitenkin sellaisenaan haavoittuvainen erillaisille tekstipohjaisille hyökkäyksille.

Tässä tutkielmassa tutustutaan näiden luonnollisen kielen käsittely historiaan, aikaisemmin mainittujen hyökkäysten mahdollistajiin sekä näiden torjuntametodeihin.

\end{abstract}
\end{otherlanguage}
