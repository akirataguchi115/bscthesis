% \begin{abstract}{finnish}

% Tämä dokumentti on tarkoitettu Helsingin yliopiston tietojenkäsittelytieteen osaston opin\-näyt\-teiden ja harjoitustöiden ulkoasun ohjeeksi ja mallipohjaksi. Ohje soveltuu kanditutkielmiin, ohjelmistotuotantoprojekteihin, seminaareihin ja maisterintutkielmiin. Tämän ohjeen lisäksi on seurattava niitä ohjeita, jotka opastavat valitsemaan kuhunkin osioon tieteellisesti kiinnostavaa, syvällisesti pohdittua sisältöä.


% Työn aihe luokitellaan  
% ACM Computing Classification System (CCS) mukaisesti, 
% ks.\ \url{https://dl.acm.org/ccs}. 
% Käytä muutamaa termipolkua (1--3), jotka alkavat juuritermistä ja joissa polun tarkentuvat luokat erotetaan toisistaan oikealle osoittavalla nuolella.

% \end{abstract}

\begin{otherlanguage}{finnish}
\begin{abstract}
Tekstin automaattisella luokituksella on tärkeä rooli digiyhteiskunnassa. Tämä luokitus tapahtuu luonnollisen kielen käsittelyyn pohjautuvilla metodeilla. Metodeihin pohjautuvat malli ovat kuitenkin haavoittuvaisia, ja tässä tutkimuksessa käsitelläänkin tekstipohjaisia vastakkaishyökkäyksiä NLP-malleja vastaan. Tutkielman alussa perehdytään automaattisen luokituksen käyttötarkoituksiin. Tämän jälkeen tutustutaan hyökkäystaksonomiaan, eli erilaisiin hyökkäysmetodeihin NLP-malleja vastaan. Lopuksi käsitellään puolustusmetodeita NLP-hyökkäyksiä vastaan.

\end{abstract}
\end{otherlanguage}
