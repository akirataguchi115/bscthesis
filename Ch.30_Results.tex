\chapter{Hyökkäystyypit\label{results}}

NLP-luokittimia vastaan, jotka ovat automaattisen tekstin luokituksen keskiössä, voidaan hyökätä. Tässä kappaleessa käydään läpi hyökkäystyypit NLP-luokittimia vastaan. Ensin käydään läpi roskapostisuodatuksen roskapostisuodatuksen ohitus, joka on NLP-hyökkäysten keskiössä. Sitten esitellään sensuuriohitus sekä ladontahyökkäykset. Ladontahyökkäyksistä käydään läpi näkymättömät merkit, homoglyfit, uudelleenjärjestelyt sekä poistatukset.

\section{Roskapostin naamiointi asiapostiksi}
Vastakkaishyökkäyksiä voidaan käyttää sähköposteissa roskapostisuodattimien ohitukseen. Roskapostisuodattimet toimivat koulutettujen NLP-luokittimien mukaan. Nämä luokittimet luokittelevat vastaanotetut sähköpostit joko hyväntahtoisiksi tai pahantahtoisiksi, eli roskaposteiksi \citep{spamfilter}.

Suodattimia vastaan toimii kolme vastakkaishyökkäystä. (1) Synonyymin korvaus. Synonyymin korvauksessa tarkoitus on korvata pahantahtoiset sanat hyväntahtoisiksi luokitelluilla synonyymeillä. Lauseiden samankaltaisuuksien vertailua demonstroidaan taulukossa 3.1. Ensimmäisessä sarakkeessa on viesti, jonka luokitin luokittelee joko roskapostiksi tai asiapostiksi toisessa sarakkeessa. Taulukon viimeinen rivi demonstroi synonyymien kovauksen läpäisevän roskapostisuodattimen. Pahantahtoisissa lauseissa pyritään nostattamaan samankaltaisuusastetta vaihtamalla sanoja synonyymeihin, kunnes NLP-luokitin tunnistaa viestin olevan asiapostia. (2) asiasanan injektointi. Asiasanan injektoinnissa asiasanoja lisätään sähköpostiin niin paljon, kunnes NLP-luokitin tunnistaa roskapostin olevan asiapostia. Sana ''asiaposti'' tarkoittaa tässä yhteydessä tekstiä, jonka roskapostisuodatin on luokitellut hyväntahtoiseksi. Asiasanoja voidaan injektoida tietokannoista roskaposteihin muuttamatta viestin tarkoitusta rajusti. (3) roskapostisanojen väljennys. Roskapostisanojen väljennyksessä roskapostisanoihin sisällytetään välilyöntejä, jotta NLP-luokitin ei tunnistaisi näitä sanoja roskasanoiksi. Kun väljennystä on harjoitettu tarpeeksi, muuttuu roskaposti NLP-luokittimen näkökulmasta asiapostiksi. \citep{spamfilter}

Asiasanan injektoinnille ja roskasanojen väljennykselle on olemassa erilaisia implementaatioita. Seuraavissa alaluvuissa tutustutaan ladontapohjaisiin vastakkaishyökkäyksiin. Muun muassa näitä hyökkäysmetodeita voidaan käyttää kahdessa aiemmin mainitussa roskapostisuodattimeen kohdistetussa hyökkäyksessä. Implementaatioita yhdistelemällä ja vaihtelemalla, saattaa NLP-luokittimen pahantahtoisuuden havaitseminen heikentyä entisestään, taaten hyökkääjälle varmemman onnistumisen.

\begin{table}[t]
  \begin{tabularx}{\textwidth}{| >{\raggedright\arraybackslash}X | c |}
    \hline
    Muokattu viesti & Ennustus\\
    \hline
    Ringtone Club: Get the UK singles chart on your mobile each week and choose any top quality ringtone! This message is free of charge. & roskapostia\\
    \hline
    Ringtone Club: \textbf{acquire} the UK single \textbf{graph} on your \textbf{Mobile\_River} each \textbf{hebdomad} and \textbf{take} any \textbf{top\_side} \textbf{caliber} ringtone! This \textbf{content} is \textbf{free\_people} of charge. & roskapostia\\
    \hline
    Ringtone Club: \textbf{become} the UK \textbf{bingle} \textbf{graph} on your \textbf{nomadic} each \textbf{workweek} and \textbf{select} any \textbf{upper\_side} \textbf{caliber} ringtone! This \textbf{subject\_matter} is \textbf{liberate} of charge. & roskapostia \\
    \hline
    Ringtone Club: \textbf{go} the UK \textbf{one} \textbf{graph} on your \textbf{peregrine} each \textbf{calendar\_week} and \textbf{pick\_out} any \textbf{upside} \textbf{character} ringtone! This \textbf{substance} is \textbf{release} of charge. & asiapostia \\
    \hline
  \end{tabularx}
  \caption{Synonyymin korvaus. Vanhan viestin korvatut osat on lihavoitu. \citep{spamfilter}}
\end{table}

\section{Näkymättömät merkit}
Näkymättömät merkit vaikuttavat tietokoneen NLP-luokittimen ymmärtämään sisältöön. Seuraavissa alikappaleissa käydään läpi ladontatason vastakkaishyökkäysmetodeita. Näitä yhdistelemällä, jopa alan johtava vihapuhefiltteri Google Perspective, on altis sensuurin ohitukselle vastakkaishyökkäyskoulutuksesta huolimatta \citep{hatespeech}. Koulutus vastakkaishyökkäyksiä vastaan tässä kontekstissa tarkoittaa puolustavan NLP-luokittimen koulutusta syötteellä, joka yrittäisi hyökätä NLP-luokitinta vastaan. Kyseessä on siis NLP-luokitin, jonka luokitus lujittuu vastakkaishyökkäyksiä tuottavan NLP-luokittimen ulostulolla. Kyseinen hyökkäys perustuu Unicode-merkistöstandardiin, joka sisältää yksilöivät koodiarvot yli 100 000 kirjoitusmerkille, tähän kuuluvat myös aakkoset sekä erikoismerkit \citep{boucher2021bad}.

Esimerkki tällaisesta erikoismerkistä on nollatilavuuden välilyönti -merkki, jonka Unicode merkintä on \texttt{U+200B}. Tällä merkillä voimme esimerkiksi vaikuttaa pelichattiin lähetettävän myrkyllissuodatettavaan merkkijonoon "olet huono" niin, että merkkijono menisi NLP-luokittimen läpi chätistä. Merkkijono \texttt{ol\textcolor{red}{U+200B}et hu\textcolor{red}{U+200B}ono} saattaisi mennä läpi chatin suodattimesta, mutta vastapuolelle viesti olisi edelleen \texttt{olet huono}.

Kontekstin poistamisen lisäksi näkymättömillä merkeillä voidaan myös tuoda ja syrjäyttää konteksteja toisilla. Esimerkiksi:\\
\texttt{Mikä pyhäinhäväistyksen rakennus!\\
  Miten onnistuit tekemään tämän näin laiskasti?} -tekstin negatiivisuus voidaan\\
syrjäyttää positiivisuudella syöttämällä NLP-luokittimelle sen sijaan teksti:\\
\texttt{Mikä py\textcolor{red}{U+200B}häinhäv\textcolor{red}{U+200B}äisty\textcolor{red}{U+200B}ksen\textcolor{red}{U+200B} rakennus!\\
  Miten onnistuit tekemään tämän \textcolor{red}{U+200B}nä\textcolor{red}{U+200B}in la\textcolor{red}{U+200B}iskas\textcolor{red}{U+200B}ti?}.

Taulukko 3.2 on esimerkki kontekstin syrjäyttämisestä näkymättömillä merkeillä. Esimerkissä tapahtuu käännös englannin kielestä ranskan kieleen. Vasemassa sarakkeessa on alkuperäinen, käännettävä viesti. Vasemmassa sarakkeessa taas on käännetty viesti ranskaksi. Alleviivattuihien kohtien väliin on upotettu nollatilavuuden välilyönti-merkki \texttt{U+200B}. Ihmiselle käännettävät viestit näyttävät samanlaisilta, mutta kääntäjälle jälkimmäisessä viestissä on kolme välilyöntiä enemmän. Viimeinen viesti pitäisi kääntyä viestiksi ''Teknologia on olemassa sitä varten''. Käännös on kuitenkin hyökkäyksen jälkeen ''Ympäristön kronologia on kronologia ympäristöstä ympäristölle''.

\begin{table}[hbt]
  \begin{tabular}{| l | m{20em} |}
    \hline
    Alkuperäinen viesti & Käännös\\
    \hline
    The technology is there to do it. & La technologie est là pour le faire.\\
    \hline
    The t\underline{ec}hnology is \underline{ t}here to\underline{ d}o it. & La chnologie de l'environnement est la chnologie de l'environnement à l'environnement.\\
    \hline
  \end{tabular}
  \caption{Hyökkäys näkymättömillä merkeillä \citep{boucher2021bad}}
\end{table}

\section{Homoglyfit}
Homoglyfihyökkäykset NLP-luokittimia vastaan pohjautuvat siihen, että pahantahtoisten\\ merkkien viralliset esitysmuodot näyttäytyvät hyväntahtoisilta merkkien virallisilta esityksiltä. Jois\-sain kielissä tekstin merkitys muuttuu täysin yhden merkin vaihtuessa. Esimerkkinä homoglyfistä on \texttt{A $\rightarrow$ A}, missä viimeinen kirjain on todellisuudessa kyrillinen kirjain A. Taulkossa 3.3 homoglyfihyökkäys on muuntanut englanninkielisen tekstin\\ \texttt{I just can't belive where she was} ranskankieliseen käännökseen\\ \texttt{I guess I can't underestimate the location of the scribe and}.\\
Näkymättömien merkkien lailla homoglyfihyökkäyksen toteutus riippuu ympäristön fontista. \citep{boucher2021bad}

\begin{table}[hbt]
  \begin{tabular}{| l | m{20em} |}
    \hline
    Alkuperäinen viesti & Käännös\\
    \hline
    I just can't believe where she was. & Je ne peux tout simplement psa croire où elle était.\\
    \hline
    I \underline{j}ust can't bel\underline{i}eve where s\underline{h}e was. & Je crois que je ne peux pas sous-estimer l'endroit où se trouvait le scribe e.\\
    \hline
  \end{tabular}
  \caption{Homoglyfihyökkäys \citep{boucher2021bad}}
\end{table}

\section{Uudelleenjärjestelyt}
Uudelleenjärjestelyhyökkäys pohjautuu näennäisen tekstin uudelleenjärjestämiseen pahantahtoisesti. Pankkitilinumeron \texttt{1234567} pystyy esimerkiksi vaihtamaan kaksisuuntaisella algoritmilla tilinumeroksi \texttt{7654321} pankin palvelinpuolella maksajan huomaamatta mitään. Unicode-merkintä tälle suunnanvaihdolle on \texttt{U+200F}. Kaksisuuntaisella algoritmilla tarkoitetaan tässä Unicoden kaksisuuntaista algoritmia. Tämän algoritmin tarkoituksena asiallisissa tarkoituksissa kääntää kirjoitussuunta vasemmalle muun muassa arabiaa tai hepreaa kirjoittaessa. Uudelleenjärjestelyjä käytetään myös NLP-luokittimen sekoittamiseen, jolloin tulokset NLP-luokittimesta ovat käyttökelvottomia. Taulukossa 3.4 uudelleenjärjestelyhyökkäys merkeissä \texttt{la} aiheuttaa ranskankielisen käännöksen järjettömyyden. Tämänlaista hyökkäystä voisi käyttää digitaalista sanakirjaa tai kääntäjää vastaan. \citep{boucher2021bad} \texttt{U+200F} ladotaan näkymättömänä näkymättömien merkkien tapaan.

\begin{table}[hbt]
  \begin{tabular}{| l | m{20em} |}
    \hline
    Alkuperäinen viesti & Käännös\\
    \hline
    A black box in your car? & Une boîte noire dans votre voiture ?\\
    \hline
    A b\underline{la}ck box in your car? & A b c h a c h a c h a c h a c h a c h a c h a c h e ?\\
    \hline
  \end{tabular}
  \caption{Uudelleenjärjestelyhyökkäys \citep{boucher2021bad}}
\end{table}

\section{Poistatukset}

Viimeisenä käydään läpi poistatushyökkäykset. Poistatushyökkäyksen tarkoituksena on poistaa käyttäjälle näkyvästä tekstistä ladontavaiheessa haluttu määrä tekstiä pois. Uhri esimerkiksi voisi olla myymässä asuntoaan, jolloin hän kopioi ja liittää sähköiseen sopimukseen hyökkääjän ehdottaman summan. Latomisvaiheessa käyttäjä kuitenkin unohtaa tarkistaa sopimuksen, jolloin poistatusmerkit saattavat poistaa myyntihinnasta esimerkiksi muutaman nollan.

Poistatushyökkäyksiä on vaikeampi toteuttaa aikaisempiin metodeihin verrattuna. Tämä johtuu useimpien käyttöjärjestelmien estosta kopioida poistatusta sisältävää tekstiä leikepöydälle suoraviivaisilla tavoilla, joilla uhri sen tekisi. Onnistuakseen poistatushyökkäyksessä, hyökkääjän tarvitsee yleisesti injektoida NLP-luokittimeen poistatus itse. Esimerkkejä poistatusmerkeistä ovat askelpalautin (BS, eng. backspace), delete (DEL) sekä vaununpalautus (CR, eng. carriage return). \citep{boucher2021bad}
Kuva 3.4 havainnollistaa poistatushyökkäystä käytännössä. Esimerkissä  näkymättömiä poistatusmerkkejä on pistetty sanojen väliin, muuttaen näin ladottujen lauseiden merkityksen. Viimeinen viesti pitäisi kääntyä viestiksi ''Tämä on tosiaan pakollinen valtiollemme.''. Käännös on kuitenkin hyökkäyksen jälkeen ''Tämä todellisuus on pakollinen rakkauden syntymälle''.

\begin{table}[hbt]
  \begin{tabular}{| l | m{17em} |}
    \hline
    Alkuperäinen viesti & Käännös\\
    \hline
    This really is a must for our nation. & C'est vraiment une nécessité pour notre nation.\\
    \hline
    This rea\_lly\_\_ is a must for\_ our na\_tion. & Cette réalyya est un incontournable pour la naissance de l'amour.\\
    \hline
  \end{tabular}
  \caption{Poistatushyökkäys \citep{boucher2021bad}}
\end{table}

Vaikka NLP-luokittimia vastaan kohdistuvia hyökkäyksiä on monta ja osa hyökkäyksistä on vaikeasti havaittavia, näiltä pyritään silti suojautumaan eri menetelmillä.