\chapter{Puolustusmetodit\label{results}}
\section{OCR-puolustus}
NLP-hyökkäykset voidaan estää alhaisemmalla tasolla overheadilla sekä korkeammalla tasolla edistyneen teknologian turvin \citep{boucher2021bad}. Näytöltäluvun (eng. OCR, On-Screen-Reading) avulla epäselvyydet tekstin aidosta luonteesta voidaan uudelleenrenderöidä tulkitsemalla aineisto uudestaan visuaalisesti. Tämä metodi lisää overheadia huomattavasti riippuen käyttötarkoituksesta, mutta poistaa pahantahtoiset merkit ilman NLP-mallin uudelleenkoulutusta.

\section{Suorituskykykeskeinen puolustus}
Keskitymme seuraavaksi näkymättömiin merkkeihin, -homoglyyfeihin, \-uudelleenjärjestelyihin -ja poistatuksiin perustuvien hyökkyksien puolustamiseen. Suorituskykykeskeiset puolustusmetodit ovat kuitenkin laskennallisesti kalliita, eivätkä koneoppimismallin ulkoistaneet firmat pysty kustantamaan kyseisiä metodeita \citep{https://doi.org/10.48550/arxiv.1911.07399}.

Tietyt näkymättömät merkit voidaan poistaa suoraa syötteestä. Mikäli applikaatiossa näitä merkkejä ei voida poistaa, voidaan ne korvata non-<unk> upotuksilla.

Homoglyyfihyökkäysten torjuminen OCR-metodilla on ymmärrettävästi vaikeampaa verrattuina muihin merkkeihin. Paras keino torjua tällaisia hyökkäyksiä olisi mapata osa homoglyyfeistä niiden yleisemmin tunnettuihin vastineisiin. NLP-mallin ylläpitäjä joutuu tekemään tässä siis suurimman jalkatyön.

Uudelleenjärjestelyhyökkäykset voidaan torjua riisumalla kaksisuuntais-ohjausmerkit\\syötteestä, varoittamalla käyttäjää kaksisuuntais-ohjausmerkkien ilmestyessä syötteeseen tai käyttämällä kaksisuuntais-algoritmia halutun syötteen selvittämiseen. Puolustusmetodin valinta riippuu kontekstista, sillä esimerkiksi latinaa tai arabiaa kirjoittaessa ohjelma toimisi väärin pakottamalla käyttäjän syötteestä pois kaksisuuntais-ohjausmerkin \texttt{U+200F}.

Poistatukset yleensä havaitaan NLP-mallien ulkopuolella syötteen annon alkuvaiheessa. NLP-mallin tasolla tähän tarvitsee harvemmin puuttua ja käyttäjälle voidaan pahimmassa tapauksessa lähettää varoitus poistatusmerkkien olemassaolosta syötteessä. On silti tärkeää tiedostaa poistatuksien puolustus, mikäli käyttöjärjestelmä unohtaa puuttua kyseiseen hyökkäysrajapintaan.