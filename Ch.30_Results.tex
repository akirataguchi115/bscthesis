\chapter{Hyökkäystyypit\label{results}}

Tässä kappaleessa käydään läpi hyökkästaksonomia, eli hyökkäysrajapinta, NLP-malleja vastaan. Ensin käydään läpi roskapostisuodatuksen roskapostisuodatuksen ohitus, joka on NLP-hyökkäysten keskiössä. Sitten esittelen neuroverkkohyökkäykset, sensuuriohituksen sekä ladontahyökkäykset.

\section{Roskapostisuodatuksen ohitus}
Vastakkaishyökkäyksiä voidaan käyttää sähköposteissa roskapostisuodattimien ohitukseen. Roskapostisuodattimet toimivat koulutettujen NLP-mallien mukaan. Nämä mallit merkkaavat vastaanotetut sähköpostit joko hyväntahtoisiksi tai pahantahtoisiksi, eli roskaposteiksi \citep{spamfilter}.

Suodattimia vastaan toimii kolme vastakkaishyökkäystä: (1) Synonyymin korvaus, (2) kelposanan injektointi sekä (3) roskapostisanojen väljennys. Sana ''kelpo'' tarkoittaa tässä yhteydessä tekstiä, jonka roskapostisuodatin on merkinnyt hyväntahtoiseksi. Synonyymin korvauksessa tarkoitus on korvata pahantahtoiset sanat hyväntahtoisiksi luokitelluilla synonyymeillä. Lauseiden samankaltaisuuksien vertailua demonstroidaan taulukossa 3.1. Pahantahtoisissa lauseissa pyritään nostattamaan samankaltaisuusastetta vaihtamalla sanoja synonyymeihin, kunnes NLP-malli tunnistaa viestin olevan kelpopostia. Kelposanan injektoinnissa kelposanoja lisätään sähköpostiin niin paljon, kunnes NLP-malli tunnistaa roskapostin olevan kelpopostia. Kelposanoja voidaan injektoida tietokannoista roskaposteihin muuttamatta viestin tarkoitusta rajusti. Roskapostisanojen väljennyksessä roskapostisanoihin sisällytetään välilyöntejä, jotta NLP-malli ei tunnistaisi näitä sanoja roskasanoiksi. Kun väljennystä on harjoitettu tarpeeksi, muuttuu roskaposti NLP-mallin näkökulmasta kelpopostiksi. \citep{spamfilter}

Kelposanan injektoinnille ja roskasanojen väljennykselle on olemassa erilaisia implementaatioita. Seuraavissa alaluvuissa tutustutaan ladontapohjaisiin vastakkaishyökkäyksiin. Muun muassa näitä hyökkäysmetodeita voidaan käyttää kahdessa aiemmin mainitussa roskapostisuodattimeen kohdistetussa hyökkäyksessä. Implementaatioita yhdistelemällä ja vaihtelemalla, saattaa NLP-mallin pahantahtoisuuden havaitseminen heikentyä entisestään, taaten hyökkääjälle varmemman onnistumisen.

\begin{table}[t]
  \begin{tabularx}{\textwidth}{| >{\raggedright\arraybackslash}X | l | c |}
    \hline
    Muokattu viesti & Samankaltaisuus & Ennustus\\
    \hline
    Ringtone Club: Get the UK singles chart on your mobile each week and choose any top quality ringtone! This message is free of charge. & 1 & roskapostia\\
    \hline
    Ringtone Club: \textbf{acquire} the UK single \textbf{graph} on your \textbf{Mobile\_River} each \textbf{hebdomad} and \textbf{take} any \textbf{top\_side} \textbf{caliber} ringtone! This \textbf{content} is \textbf{free\_people} of charge. & $0,583$ & roskapostia\\
    \hline
    Ringtone Club: \textbf{become} the UK \textbf{bingle} \textbf{graph} on your \textbf{nomadic} each \textbf{workweek} and \textbf{select} any \textbf{upper\_side} \textbf{caliber} ringtone! This \textbf{subject\_matter} is \textbf{liberate} of charge. & $0,583$ & roskapostia \\
    \hline
    Ringtone Club: \textbf{go} the UK \textbf{one} \textbf{graph} on your \textbf{peregrine} each \textbf{calendar\_week} and \textbf{pick\_out} any \textbf{upside} \textbf{character} ringtone! This \textbf{substance} is \textbf{release} of charge. & $0,583$ & kelpopostia \\
    \hline
  \end{tabularx}
  \caption{Synonyymin korvaus. Vanhan viestin korvatut osat on lihavoitu. \citep{spamfilter}}
\end{table}

\section{Sensuurin ohitus}
Koska sensuuria voidaan soveltaa hyödyntäen koneoppimismalleja, voidaan sensuuri myös ohittaa hyödyntäen koneoppimismallin heikkouksia. Vastakkaishyökkäys voisi tunnistaa sensurointia aiheuttavia pikseliyhdistelmiä, ja tässä tutkimuksessa esiteltyjä hyökkäystapoja käyttäen sensuurin laukaiseminen voidaan estää. Tällöin kyseessä ei kuitenkaan enää ole puhdas merkintä (eng. clean label), sillä vastakkaishyökkäyksen todellinen tarkoitus näkyy käyttäjälle silmintarkasteltavana \citep{triggerless}. Puhtaan merkinnän uupuessa esimerkiksi tekstipohjaisessa vastakkaishyökkäyksessä mahdollissta myös helpomman puolustautumisen \citep{pruthi2019}.

\section{Näkymättömät merkit}
Näkymättömät merkit vaikuttavat tietokoneen NLP-mallin ymmärtämään sisältöön. Kyseinen hyökkäys perustuu Unicode-merkistöstandardiin, joka sisältää yksilöivät koodiarvot kirjoitushetkellä yli 100 000 kirjoitusmerkille, tähän kuuluvat myös aakkoset sekä erikoismerkit.

Esimerkki tällaisesta erikoismerkistä on nollatilavuuden välilyönti -merkki, jonka Unicode merkintä on \texttt{U+200B}. Tällä merkillä voimme esimerkiksi vaikuttaa pelichattiin lähetettävän myrkyllissuodatettavaan merkkijonoon "olet huono" niin, että merkkijono menisi NLP-mallin läpi chätistä. Merkkijono \texttt{ol\textcolor{red}{U+200B}et hu\textcolor{red}{U+200B}ono} saattaisi mennä läpi chatin suodattimesta, mutta vastapuolelle viesti olisi edelleen \texttt{olet huono}.

Kontekstin poistamisen lisäksi näkymättömillä merkeillä voidaan myös tuoda ja syrjäyttää konteksteja toisilla. Esimerkiksi:\\
\texttt{Mikä pyhäinhäväistyksen rakennus!\\
  Miten onnistuit tekemään tämän näin laiskasti?} -tekstin negatiivisuus voidaan\\
syrjäyttää positiivisuudella syöttämällä NLP-mallille sen sijaan teksti:\\
\texttt{Mikä py\textcolor{red}{U+200B}häinhäv\textcolor{red}{U+200B}äisty\textcolor{red}{U+200B}ksen\textcolor{red}{U+200B} rakennus!\\
  Miten onnistuit tekemään tämän \textcolor{red}{U+200B}nä\textcolor{red}{U+200B}in la\textcolor{red}{U+200B}iskas\textcolor{red}{U+200B}ti?}.

Kuva 3.1 on esimerkki kontekstin syrjäyttämisestä näkymättömillä merkeillä. Esimerkissä tapahtuu käännös englannin kielestä ranskan kieleen.
\begin{figure}[hbt]
  \includegraphics[scale=0.5]{figures/invisible.png}
  \caption{Hyökkäys näkymättömillä merkeillä \citep{boucher2021bad}}
\end{figure}



\section{Homoglyfit}
Homoglyfihyökkäykset NLP-malleja vastaan pohjautuvat siihen, että pahantahtoisten\\ merkkien viralliset esitysmuodot näyttäytyvät hyväntahtoisilta merkkien virallisilta esityksiltä. Jois\-sain kielissä tekstin merkitys muuttuu täysin yhden merkin vaihtuessa. Esimerkkinä homoglyfistä on \texttt{A $\rightarrow$ A}, missä viimeinen kirjain on todellisuudessa kyrillinen kirjain A. Kuvassa 3.3 homoglyfihyökkäys on muuntanut englanninkielisen tekstin\\ \texttt{I just can't belive where she was} ranskankieliseen käännökseen\\ \texttt{I guess I can't underestimate the location of the scribe and}.
\begin{figure}[hbt]
  \includegraphics[scale=0.5]{figures/homoglyph.png}
  \caption{Homoglyfihyökkäys \citep{boucher2021bad}}
\end{figure}
\\Näkymättömien merkkien lailla homoglyfihyökkäyksen toteutus riippuu ympäristön fontista. \citep{boucher2021bad}


\section{Uudelleenjärjestelyt}
Uudelleenjärjestelyhyökkäys pohjautuu näennäisen tekstin uudelleenjärjestämiseen pahantahtoisesti. Pankkitilinumeron \texttt{1234567} pystyy esimerkiksi vaihtamaan kaksisuuntaisella-algoritmilla tilinumeroksi \texttt{7654321} pankin palvelinpuolella maksajan huomaamatta mitään. Unicode-merkintä tälle suunnanvaihdolle on \texttt{U+200F}. Uudelleenjärjestelyjä käytetään myös NLP-mallin sekoittamiseen, jolloin tulokset NLP-mallista ovat käyttökelvottomia. Kuvassa 3.4 uudelleenjärjestelyhyökkäys merkeissä \texttt{la} aiheuttaa ranskankielisen käännöksen järjettömyyden. Tämänlaista hyökkäystä voisi käyttää digitaalista sanakirjaa tai kääntäjää vastaan. \citep{boucher2021bad}
\begin{figure}[t]
  \includegraphics[scale=0.5]{figures/reordering.png}
  \caption{Homoglyfihyökkäys \citep{boucher2021bad}}
\end{figure}
\texttt{U+200F} ladotaan näkymättömänä näkymättömien merkkien tapaan.

\section{Poistatukset}

Viimeisenä käydään läpi poistatushyökkäykset. Poistatushyökkäyksiä on vaikeampi toteuttaa aikaisempiin metodeihin verrattuna. Tämä johtuu useimpien käyttöjärjestelmien estosta kopioida poistatusta sisältävää tekstiä leikepöydälle suoraviivaisilla tavoilla, joilla uhri sen tekisi. Onnistuakseen poistatushyökkäyksessä, hyökkääjän tarvitsee yleisesti injektoida NLP-malliin poistatus itse. Esimerkkejä poistatusmerkeistä ovat askelpalautin (BS, eng. backspace), delete (DEL) sekä vaununpalautus (CR, eng. carriage return). \citep{boucher2021bad}
Kuva 3.5 havainnollistaa poistatushyökkäystä käytännössä.
\begin{figure}[t]
  \includegraphics[scale=0.5]{figures/backspace.png}
  \caption{Poistatushyökkäys \citep{boucher2021bad}}
\end{figure}