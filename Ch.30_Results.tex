\chapter{Puolustusmetodit\label{results}}
\section{OCR-puolustus}
NLP-hyökkäykset voidaan estää alhaisemmalla tasolla overheadilla sekä korkeammalla tasolla edistyneen teknologian turvin \citep{boucher2021bad}. Näytöltäluvun (eng. OCR, On-Screen-Reading) avulla epäselvyydet tekstin aidosta luonteesta voidaan uudelleenrenderöidä tulkitsemalla aineisto uudestaan visuaalisesti. Tämä metodi lisää overheadia huomattavasti riippuen käyttötarkoituksesta, mutta poistaa pahantahtoiset merkit ilman NLP-mallin uudelleenkoulutusta.

\section{Suorituskykykeskeinen puolustus}
Keskitymme seuraavaksi näkymättömiin merkkeihin, -homoglyyfeihin, \-uudelleenjärjestelyihin -ja poistatuksiin perustuvien hyökkyksien puolustamiseen.

Tietyt näkymättömät merkit voidaan poistaa suoraa syötteestä. Mikäli applikaatiossa näitä merkkejä ei voida poistaa, voidaan ne korvata non-<unk> upotuksilla.

Homoglyyfihyökkäysten torjuminen OCR-metodilla on ymmärrettävästi vaikeampaa verrattuina muihin merkkeihin. Paras keino torjua tällaisia hyökkäyksiä olisi mapata osa homoglyyfeistä niiden yleisemmin tunnettuihin vastineisiin. NLP-mallin ylläpitäjä joutuu tekemään tässä siis suurimman jalkatyön.

Uudelleenjärjestelyhyökkäykset voidaan torjua riisumalla bidi-ohjausmerkit syötteestä, varoittamalla käyttäjää bidi-ohjausmerkkien ilmestyessä syötteeseen tai käyttämällä bidi-algoritmia halutun syötteen selvittämiseen.

Poistatukset yleensä havaitaan NLP-mallien ulkopuolella syötteen annon ensivaihessa. NLP-mallin tasolla tähän tarvitsee harvemmin puuttua ja käyttäjälle voidaan pahimmassa tapauksessa lähettää varoitus poistatusmerkkien olemassaolosta syötteessä.