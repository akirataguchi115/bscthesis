\chapter{Puolustusmenetelmät\label{results}}
Käydään läpi puolustusmenetelmät NLP-hyökkäyksiä vastaan.

\section{OCR-puolustus}
NLP-hyökkäykset voidaan torjua korkean tason abstraktiolla korkealla yleisrasituksella sekä alemman tason abstraktiolla alemmalla yleisrasituksella. Näytöltäluvun (eng. OCR, On-Screen-Reading) avulla epäselvyydet tekstin aidosta luonteesta voidaan hahmontaa uudelleen tulkitsemalla aineisto uudestaan visuaalisesti. Tämä menetelmä lisää yleisrasitusta huomattavasti riippuen käyttötarkoituksesta, mutta poistaa pahantahtoiset merkit ilman NLP-mallin uudelleenkoulutusta. \citep{boucher2021bad}

\section{Suorituskykykeskeinen puolustus}
Keskitymme seuraavaksi näkymättömiin merkkeihin, -homoglyfeihin, \-uudelleenjärjestelyihin -ja poistatuksiin perustuvien hyökkyksien puolustamiseen. Suorituskykykeskeiset puolustusmenetelmät ovat kuitenkin laskennallisesti kalliita, eivätkä koneoppimismallin ulkoistaneet yritykset pysty kustantamaan kyseisiä metodeita \citep{https://doi.org/10.48550/arxiv.1911.07399}.

Tietyt näkymättömät merkit voidaan poistaa suoraan syötteestä. Mikäli sovelluksessa näitä merkkejä ei voida poistaa, voidaan ne korvata \textit{ei-}\texttt{<unk>} upotuksilla. Korvaus tapahtuu lähdekielisanakirjassa, jonne kuvataan tuntematon merkki ''ei-tuntemattomaksi tokeniksi''. Näin tuntemattomat merkit eivät voi häiritä ladontaa merkeillä, joista ladontamoottori ei ole aivan varma. \citep{boucher2021bad}

Homoglyfihyökkäysten torjuminen OCR-menetelmällä on ymmärrettävästi vaikeampaa verrattuina muihin merkkeihin. Paras keino torjua tällaisia hyökkäyksiä olisi kuvata osa homoglyfeistä niiden yleisemmin tunnettuihin vastineisiin. NLP-mallin ylläpitäjä joutuu tekemään tässä siis suurimman työn. \citep{boucher2021bad}

Uudelleenjärjestelyhyökkäykset voidaan torjua riisumalla kaksisuuntais-ohjausmerkit\\syötteestä, varoittamalla käyttäjää kaksisuuntais-ohjausmerkkien ilmestyessä syötteeseen tai käyttämällä kaksisuuntais-algoritmia halutun syötteen selvittämiseen. Puolustusmenetelmän valinta riippuu kontekstista, sillä esimerkiksi latinaa tai arabiaa kirjoittaessa ohjelma toimisi väärin pakottamalla käyttäjän syötteestä pois kaksisuuntais-ohjausmerkin \texttt{U+200F}. \citep{boucher2021bad}

Poistatukset yleensä havaitaan NLP-mallien ulkopuolella syötteen annon alkuvaiheessa. NLP-mallin tasolla tähän tarvitsee harvemmin puuttua ja käyttäjälle voidaan pahimmassa tapauksessa lähettää varoitus poistatusmerkkien olemassaolosta syötteessä. On silti tärkeää tiedostaa poistatuksien puolustus, mikäli käyttöjärjestelmä unohtaa puuttua kyseiseen hyökkäysrajapintaan. \citep{boucher2021bad}

\section{Lujitus käyttäen lineaarisia luokittimia}
NLP-malleja voidaan lujittaa käyttämällä tukivektorikonetta, mikä perustuu lineaarisiin luokittimiin. Xupeng Shi ja A. Adam Ding käsittelevät tukivektorikoneen lujitusmahdollisuuksia tuktimuksessaan \textit{Understanding and Quantifying Adversarial Examples Existence in Linear Classification}. Tutkielma pohjautuu kahteen olennaiseen määritelmään:

\theoremstyle{definition}
\newtheorem{definition}{Määritelmä}

\begin{definition}
  Olkoon luokitin $C$. Datavektorin $x$ vastakkaisesimerkki $\varepsilon$ on toinen datavektori $x'$ niin, että $||x-x'||\leq \varepsilon$, mutta $C(x) \not = C(x')$.
\end{definition}

\begin{definition}
  Olkoon luokitin $C$. Datavektorin $x$ vahva vastakkaisesimerkki $(\varepsilon, \delta)$ on toinen datavektori $x'$ niin, että $||x-x'||\leq \varepsilon$ ja $|(x-x')\cdot \mu|\leq\delta$, mutta $C(x) \not = C(x')$.
\end{definition}

Tutkielmassa ehdotetaan puolustusmetodeita perustuen ylläoleviin määritelmiin sekä niiden sovelluksiin käyttäen lineaarisia luokittimia.